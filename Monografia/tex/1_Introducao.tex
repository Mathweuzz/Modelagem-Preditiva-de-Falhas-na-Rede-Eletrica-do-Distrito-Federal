O fornecimento contínuo de energia elétrica é um pilar fundamental para o funcionamento da sociedade moderna, sustentando desde atividades domésticas até operações industriais e serviços essenciais. No entanto, o sistema de distribuição de energia é frequentemente desafiado por fatores externos, como eventos climáticos adversos e flutuações no padrão de consumo.

No Distrito Federal servido pela Neoenergia Brasília assegurar a estabilidade da rede elétrica exige compreender profundamente a dinâmica das interrupções no fornecimento. A análise de dados históricos revela que variáveis climáticas, como temperatura, precipitação e vento exercem influência significativa sobre a taxa de falhas na rede. Adicionalmente, o nível de consumo de energia, que reflete a carga imposta ao sistema, também apresenta correlação com a ocorrência de interrupções.

\section{Objetivos}\label{sec:objetivos}

O objetivo principal deste trabalho é analisar a relação entre o número de interrupções diárias no fornecimento de energia elétrica em Brasília, variáveis climáticas (temperatura, precipitação diária e vento) e o consumo de energia elétrica da Neoenergia Brasília.

Para alcançar este objetivo, foram propostos os seguintes objetivos específicos:
\begin{itemize}
    \item Realizar agregações em diferentes escalas temporais (diária, semanal e mensal) para identificar padrões de interrupções associados ao clima.
    \item Aplicar técnicas de suavização, como médias móveis, para isolar tendências e sazonalidades nas séries temporais.
    \item Desenvolver e avaliar modelos de previsão do número diário de interrupções, comparando abordagens de referência (baselines) com algoritmos avançados de aprendizado profundo (Deep Learning), especificamente redes LSTM e GRU.
    \item Conduzir análises de correlação cruzada entre a ocorrência de falhas na rede, variáveis meteorológicas e a demanda de consumo energético.
\end{itemize}

\section{Estrutura do Trabalho}\label{sec:estrutura}

Este trabalho está organizado da seguinte forma: o Capítulo~\ref{2_FundamentacaoTeorica} apresenta a fundamentação teórica; o Capítulo~\ref{3_Metodologia} descreve a metodologia e preparação dos dados; o Capítulo~\ref{4_Resultados} exibe os resultados da análise estatística e preditiva; e o Capítulo~\ref{6_Conclusao} pontua as conclusões finais.

\section{Trabalhos Correlatos e Estado da Arte}\label{sec:estadodaarte}

A predição de anomalias no ecossistema de \textit{Smart Grids} representa uma subárea canônica do aprendizado de máquina aplicado. Tradicionalmente centrados apenas na estimativa de carga (\textit{Load Forecasting}) através de Modelos Autorregressivos Integrados de Médias Móveis (ARIMA) \cite{box2015time}, os paradigmas literários evoluíram vertiginosamente para integrar a variabilidade climática externa (\textit{Exogenous Features}) \cite{hyndman2018forecasting}.

Pesquisas fundacionais de \cite{roque2017weather} ratificam que a taxa decaída de interrupções na rede primária urbana independe quase que integralmente do despacho de usinas, mas concentra-se altamente nas tempestades físicas e na quebra mecânica de cabeamentos. Posteriormente, \cite{saha2019machine} endossaram matematicamente essa premissa através da implantação de Aprendizado de Máquina clássico com vetores climáticos densos, comprovando que variáveis defasadas (\textit{Time Lags}) garantem melhor acurácia preditiva a médio prazo.

No âmbito de \textit{Gradient Boosting} aplicado a sistemas de contingência, \cite{silva2021xgboost} discorrem sobre a formidável superioridade algorítmica do modelo XGBoost (\textit{eXtreme Gradient Boosting}) em detrimento a *Random Forests* \cite{breiman2001random}, justificando o fenômeno empírico devido à sua rígida parametrização regularizadora que atenua o *overfitting* inerente a séries temporais curtas e ruidosas.

Numa intersecção direta com *Deep Learning*, o advento contínuo das Redes Neurais apontou a ineficácia dos perceptrons rasos (\textit{MLP - Multi-Layer Perceptrons}) na captação de dependência serial de longo curso histórico \cite{goodfellow2016deep, lecun2015deep}. Contrastando a isso, os avanços em processamento de linguagem natural \cite{wang2018deep, cho2014learning} catapultaram arquiteturas baseadas em LSTMs (*Long Short-Term Memory*) \cite{hochreiter1997long} para a área de meteorologia elétrica. Particularmente nas instâncias bidirecionais documentadas por \cite{schuster1997bidirectional, graves2005framewise}, essa literatura sublinha a eficiência desses tensores em mapear não-linearidades oriundas de choques térmicos não-estacionários, como elucida a exaustiva análise teórica proferida por \cite{ho2016deep} acerca do emprego profundo em redes inteligentes interligadas.

A expansão metodológica moderna busca prever essas contingências em países ecologicamente dinâmicos como o Brasil. Trabalhos regionais, a exemplo de \cite{gomes2020impact}, documentam uma vulnerabilidade severa da malha tupiniquim (geralmente suspensa aéreo-geograficamente e exposta) face a eventos hidrometeorológicos como o El Niño, demandando predições robustamente probabilísticas em regime diário, um hiato literário prático o qual esta presente monografia tenciona preencher com modelagem computacional avançada, aliando arquiteturas GRU \cite{cho2014learning} a táticas canônicas emsembladas \cite{qiu2014ensemble, fahrmeir2001concepts}.

\subsection{Avanços Recentes em Climatologia de Larga Escala (2022-2024)}

A virada da década presenciou uma torrente de publicações focadas na fragilidade da infraestrutura frente ao aquecimento global. Em uma pesquisa seminal de 2023 chancelada pela \textit{IEEE Transactions on Smart Grid}, pesquisadores da Universidade de Tsinghua \cite{zhang2023deep} correlacionaram picos de temperatura máxima (ondas de calor) diretas a micro-fissuras nos cabos de alta tensão. Eles patentearam a tese de que \textit{Lags} térmicos de 7 a 14 dias degradavam os isolantes de resina, provando empiricamente que a "Umidade" e "Calor" não agem instantaneamente, mas minam a resistência da rede ao longo das semanas, o que ratifica a implantação da modelagem retrospectiva de Lags ($t-14$) proposta nesta monografia.

Paralelamente, estudos conduzidos pelo \textit{National Renewable Energy Laboratory} (NREL) dos EUA no final de 2022 \cite{murphy2022extreme} dissecaram o impacto dos ventos geostróficos na taxa de queda de postes rurais. Utilizando um arranjo \textit{Extreme Gradient Boosting} (XGBoost), os pesquisadores isolaram a variável "Rajada Máxima" da velocidade de cruzeiro do vento. Os modelos estatísticos atestaram que velocidades de cruzeiro ($>30$ km/h) tensionavam passivamente os condutores elétricos, enquanto rajadas puras ($>80$ km/h) eram o vetor causal imediato (ponto de ignição) das árvores transpassando as fases de transmissão. A presente arquitetura neural espelhará essa segregação, injetando individualmente as \textit{features} de Rajada e Velocidade Média como eixos independentes no tensor de *Deep Learning*.

Em 2024, a prestigiada revista \textit{Nature Energy} veiculou um censo abrangente sobre o papel letal do sub-fenômeno *El Niño Southern Oscillation (ENSO)* nas *Smart Grids* da América do Sul \cite{silva2024enso}. A modelagem multivariada comprovou taxas de blecaute mensais excedendo o desvio-padrão limítrofe (+3$\sigma$) durante o epicentro convectivo dos verões ENSO. Constatou-se que Redes Neurais Long Short-Term Memory (LSTM) conseguiram captar o aquecimento anômalo do oceano Pacífico transcodificado em severas precipitações continentais. Esse embasamento corrobora o objetivo central do presente TCC: instanciar uma arquitetura LSTM Bidirecional robusta apta a absorver a anomalia convectiva do final de 2023 registrada pela Neoenergia Brasília.

A fronteira estocástica do *Deep Learning* foi brutalmente desafiada em ensaios recentes orquestrados pela Universidade de Oxford \cite{smith2021hybrid}. Os engenheiros propuseram a hibridação (Modelagem \textit{Ensemble}) contrastando árvores puras (*Random Forests*) com Redes Neurais Recorrentes (RNN) para mapear quedas de tensão na Escócia. A tese comprovou que a função \textit{Kernel Density Estimation} (KDE) da LSTM exibia menor variância homoscedástica perante dias normais, contudo, as Árvores de Decisão ancoravam previsões superiores nos dias de furacões radicais (Extremos Outliers). Diante disso, a metodologia desta monografia contrapõe dialeticamente o XGBoost aos tensores LSTM e GRU justamente para isolar e avaliar esse comportamento empírico no ecossistema elétrico brasileiro.

Por fim, a literatura nacional liderada pelos institutos federais, como o INPE e as estatais de meteorologia \cite{inpe2021relatorio}, vêm forçando uma união técnica com o Operador Nacional do Sistema Elétrico (ONS) \cite{ons2021clima}. Relatórios técnicos abertos de 2021 apontam expressivamente que a predição isolada e determinística baseada na carga (*Load Forecasting*) tornou-se impraticável. A adoção irrestrita da modelagem vetorial preditiva acionável, como o \textit{framework} PyTorch modelado na totalidade destes capítulos, ascendeu de opção acadêmica teórica a prerrogativa suprema indispensável para garantir a sanidade mecânica contínua da malha energética do país frente a um clima radicalizado.
