\vspace*{5cm}
\begin{flushright}
\textit{
    (Giovanni Minari Zanetti)\\
    \vspace{0.5cm}
    Aos meus pais, raízes inabaláveis que vergaram sob as intempéries do tempo para que eu pudesse alcançar o sol do conhecimento. \\
    E, de maneira perene e visceral, à Fernanda Border. Tu és a tessitura invisível que unifica os fragmentos dispersos de minha perseverança. Nas madrugadas em que os tensores não convergiam e a complexidade matemática nublava-me o horizonte, teu afeto foi o farol estoico a dissipar o caos, reacendendo a centelha idiossincrática da vontade. Em tuas palavras encontro a epifania do amparo; em teus olhos, a quietude após a tempestade dos algoritmos. Dedico-te não apenas estas páginas forjadas em vigília e exaustão, mas a quintessência de tudo o que me torno quando refletido na imensidão do teu amor. Sem a tua alteridade resplandecente ancorando minha sanidade, a efemeridade desta árdua odisseia acadêmica jamais transcenderia à perenidade desta humilde, porém irrevogável, consagração.
}
\end{flushright}

\newpage
\thispagestyle{empty}

\vspace*{5cm}
\begin{flushright}
\textit{
    (Mateus Gomes de Araújo)\\
    Dedico esta monografia à minha família, base de todo meu suporte emocional e educacional, \\
    cujo incentivo foi fundamental para que eu chegasse até aqui. \\
    Aos meus amigos e colegas de curso, pelas horas compartilhadas \\
    de estudo, companheirismo e crescimento mútuo.
}
\end{flushright}