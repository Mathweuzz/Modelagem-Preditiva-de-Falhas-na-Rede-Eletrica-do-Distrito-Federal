A crescente complexidade e suscetibilidade das redes de distribuição de energia elétrica frente a severos eventos climáticos adversos exigem a transição de um paradigma de manutenção reativa para uma gestão preditiva inteligente. Tradicionalmente, as concessionárias de energia confiam em despachos emergenciais pós-falha para conter os blecautes deflagrados por descargas atmosféricas e rajadas de vento, o que resulta em punições regulatórias severas baseadas nos indicadores DEC e FEC, além do prejuízo macroeconômico latente infligido à malha metropolitana.

Neste cenário de mitigação de risco e otimização da resiliência eletromecânica, este trabalho de conclusão de curso propõe uma extensa análise preditiva unificada à modelagem estatística profunda das interrupções de fornecimento (desarmes de transformadores e rompimento de cabos) no Distrito Federal (DF). A originalidade do método jaz na correlação íntima destes distúrbios operativos isolados contra as anomalias termodinâmicas climáticas locais e contra o balanço comportamental de carga da população (Picos de Demanda). 

Para construir o laboratório empírico irrestrito, amalgamou-se um vetor intercedido de 3.073 dias ininterruptos (mais de oito anos, englobando cronologicamente os períodos fáticos de 01/01/2016 a 31/05/2024). Essa volumetria temporal massiva foi arquitetada via fusão ativa (\textit{web-scraping} e consumo de relatórios abertos) de dois repositórios governamentais brasileiros distintos: (i) as matrizes climáticas telemetradas de alta resolução procedentes da estação meteorológica principal do Instituto Nacional de Meteorologia (INMET - A001); e (ii) os bancos de \textit{Data Warehouse} abertos auditados e punitivos da Agência Nacional de Energia Elétrica (ANEEL), abrangendo tanto o conjunto pericial de ocorrências elétricas puras quanto o Balanço Energético Mensal Sistêmico (SAMP).

Para expurgar as patologias randômicas temporais inerentes à coleta mecânica falha de sensores e contornar a latência estocástica em eventos meteorológicos complexos, estabeleceu-se um rigoroso e extenso ecossistema orquestrado de Engenharia de Dados de Série Temporal (\textit{Time-Series Feature Engineering}). Construiu-se mais de 40 covariáveis (\textit{Features} contínuas) derivadas através de metodologias exatas de interpolação linear para \textit{Missing Values}, cálculos geométricos estacionários da Média Móvel Exponencial (EMA) até defasagens estruturais amplas ($T-14$ Lags de latência de desgaste acumulado), além do isolamento circular do ano através de incorporações trigonométricas harmônicas de cosseno e seno sazonal. Adicionalmente, mitigou-se a maldição implícita da alta dimensionalidade aplicando a métrica exploratória termodinâmica da correlação de \textit{Pearson} sob matrizes estatísticas em mapas de calor.

A topografia metodológica, orientada empiricamente para o aprendizado de arquiteturas hierárquicas preditivas, avaliou dialeticamente e comparou a performance convergencial do algoritmo canônico puramente baseado em florestas de decisão ortogonal --- \textit{eXtreme Gradient Boosting} (XGBoost), penalizado via Regularizações Lasso / Ridge --- rotulado frente ao estado da arte em propagação diferencial para grafos acíclicos em Deep Learning: as Redes Neurais Profundas de Recorrência (\textit{Recurrent Neural Networks}). Emulou-se em tensores as \textit{Long Short-Term Memory} (Bi-LSTM) e \textit{Gated Recurrent Unit} (Bi-GRU) paralelizadas através de processamento CUDA nativo pelo pacote genérico matemático \textit{PyTorch}. 

Todos os vetores foram submetidos a uma rigorosa e agressiva separação estocástica com expurgo cronológico direcional na divisão dimensional (\textit{Time-Series Split} focado em partições \textit{Out-Of-Sample}), vetando integralmente o vazamento antecipado de dados furtivos (\textit{Data Leakage}). Os achados comprovam estatisticamente, baseados em métricas restritivas de penalização absoluta quadrática (RMSE e MAE), que a inserção sistemática de covariáveis heterogêneas associando entropia climática exógena aos ciclos contínuos de carga elétrica diurna catalisa severamente o hiper-plano preditivo $R^2$. Redes recursivas provaram ser vastamente superiores em blindagem adaptativa perante os contínuos picos meteorológicos instáveis catalizados por anomalias macro (Vórtices Climáticos de um \textit{El Niño} extremo). Finalmente, orienta-se e propõe-se a premissa sistêmica de refinar iterativamente a abstração e expatriação temporal dos algoritmos validados abrindo novas fronteiras geográficas cobrindo demais esferas sazonais isoladas dos bioclimas brasileiros nos próximos trabalhos analíticos.